% !Mode:: "TeX:UTF-8"
%!TEX program  = xelatex

\documentclass[bwprint]{cumcmthesis} 

\title{创意平板折叠桌}
\tihao{A}
\baominghao{201809002232}
\schoolname{上海交通大学}
\membera{钱文涛}
\memberb{盛志耀}
\memberc{王思极}
\supervisor{某老师}
\yearinput{2018}
\monthinput{9}
\dayinput{14}
\begin{document}
 \maketitle
 \begin{abstract}
    \indent 人们在进行高温作业时需要穿着专业的高温工作服以防危险。通常的高温工作服由三层织物材料构成,本文将研究高温工作服的隔层材料的导热性质,建立数学模型模拟来确定假人皮肤外侧的温度变化情况。\\
    \indent 针对问题一,首先建立系统的热传导方程,系统包括:人体、高温工作服以及外界热源。再考虑到人体表面接触高温工作服,高温工作服同时与外界热源接触,所以不妨将我们的方程简化成为一维热传导方程。\\
    \indent 在建立完一维热传导方程之后,结合已知的边界条件,就可以通过求解微分方程得到系统的温度分布。再将已经求解得到的温度分布与“附件2. 假人皮肤外侧的测量温度”进行比对,如果对比结果直接满足,那可以继续进行问题二、三的求解;反之,则需要对建立的模型进行修正。\\
    \indent 最后经过对比,发现数据的拟合程度确实存在一定的出入,故考虑进行修正。修正一:在工作服表面吸附有一层薄空气层,记作V层;修正二:人体内环境和工作服之间存在人体组织,记作0层。\\
    \indent 修正之后,再度求解微分方程。最终发现此时的修正结果与附件所给数据符合得很好。\\
    \indent 针对问题二,结合已经求解得到的温度分布模型,修改初始条件为:环境温度为$65^{\circ}C$、IV层的厚度为$5.5mm$,确定出合适的$\uppercase\expandafter{\romannumeral2}$层厚度,使得确保工作60分钟时,假人皮肤外侧温度不超过$47^{\circ}C$,且超过$44^{\circ}C$的时间不超过5分钟。\\
    \indent 针对问题三,结合已经求解得到的温度分布模型,修改初始条件为:环境温度为$80^{\circ}C$,确定出合适的$\uppercase\expandafter{\romannumeral2}$层和$\uppercase\expandafter{\romannumeral4}$层厚度,使得确保工作60分钟时,假人皮肤外侧温度不超过$47^{\circ}C$,且超过$44^{\circ}C$的时间不超过5分钟。\\
    \indent 最后再总体考虑所以材料的参数进行微修正,并做一定的回归分析,控制模型的误差。
\keywords{热传导方程 \quad 温度分布 \quad 误差分析 }
\end{abstract}

\section{问题的提出}
    \indent 实际生活中从事高温环境工作的人员需要穿着专业的高温工作服以防被烧伤。那么根据现有材料的相关参数模拟得出合适的服装设计就十分的重要。另外,热传导方程是一个我们熟知的偏微分方程,它具体描述了一个区域内的温度如何随时间变化。\\

\section{问题重述}
    \indent 现根据附件所提供的专用服装材料的参数值以及假人皮肤外侧的测量温度数据,建立数学模型,解决以下问题:
    \begin{enumerate}
        \item 利用附件problem1.xlsx中的相关材料参数对附件中的家人皮肤测量温度进行模拟,建立模型,计算温度分布,并生成温度分布文件。
        \item 利用在问题一中提出的模型,当环境温度为$65^{\circ}C$、IV层的厚度为$5.5mm$时,确定II层的最优厚度,确保工作60分钟时,假人皮肤外侧温度不超过$47^{\circ}C$,且超过$44^{\circ}C$的时间不超过5分钟。
        \item 利用在问题一种提出的模型,当环境温度为$80^{\circ}C$时,确定$\uppercase\expandafter{\romannumeral2}$层和$\uppercase\expandafter{\romannumeral4}$层的最优厚度,确保工作30分钟时,假人皮肤外侧温度不超过$47^{\circ}C$,且超过$44^{\circ}C$的时间不超过5分钟。
        
    \end{enumerate}

    \section{模型的假设}
    \begin{itemize}
    \item 从一维情况考虑。
    \item 不考虑热辐射和对。
    \item 假设75是一个稳定热源的温度,它和外层织物存在一个空气层,那一层的初始温度是75度。
    \item 人的皮肤和内部温度存在一层组织充当热传导材料。
    \item 人体组织和所有衣物以及衣物和人体间的空气初始温度都是$37^{\circ}C$。
    \item 所有过程满足热传导方程。
    \end{itemize}

\section{符号说明}
\begin{tabular}{cc}
 \hline
 \makebox[0.4\textwidth][c]{符号}	&  \makebox[0.5\textwidth][c]{意义} \\ \hline
 u      & 温度($^{\circ}$C) \\ \hline
 s	    & 厚度(cm)  \\ \hline
 t	    & 时间(second)  \\ \hline
 $T_{1}$	& 人体内部温度($37^{\circ}$C) \\ \hline
 $T_{2}$	& 理想外部热源温度($75^{\circ}$C) \\ \hline
 k	    &  热扩散率($m^{2}/s$)) \\ \hline
 $\rho$	    & 密度($kg/m^{2}$))  \\ \hline
 C	    & 比热($J/(kg\cdot^{\circ}C)$)  \\ \hline
 $\lambda$	    & 热传导率($W/(m\cdot^{\circ}C)$  \\ \hline
\end{tabular}

\newpage
\section{问题分析}
    \subsection{问题一分析}
        \indent 针对问题一,首先建立系统的热传导方程,系统包括:人体、高温工作服以及外界热源。再考虑到人体表面接触高温工作服,高温工作服同时与外界热源接触,所以不妨将我们的方程简化成为一维热传导方程。\\
        \indent 在建立完一维热传导方程之后,结合已知的边界条件,就可以通过求解微分方程得到系统的温度分布。再将已经求解得到的温度分布与“附件2. 假人皮肤外侧的测量温度”进行比对,如果对比结果直接满足,那可以继续进行问题二、三的求解;反之,则需要对建立的模型进行修正。\\
        \indent 最后经过对比,发现数据的拟合程度确实存在一定的出入,故考虑进行修正。修正一:在工作服表面吸附有一层薄空气层,记作$\uppercase\expandafter{\romannumeral5}$层;修正二:人体内环境和工作服之间存在人体组织,记作0层。\\
        \indent 修正之后,再度求解微分方程。最终发现此时的修正结果与附件所给数据符合得很好。
    \subsection{问题二分析}
        \indent 结合已经求解得到的温度分布模型,修改初始条件为:环境温度为$65^{\circ}C$、IV层的厚度为$5.5mm$,确定出合适的$\uppercase\expandafter{\romannumeral2}$层厚度,使得确保工作60分钟时,假人皮肤外侧温度不超过$47^{\circ}C$,且超过$44^{\circ}C$的时间不超过5分钟。
    \subsection{问题三分析}
        \indent 结合已经求解得到的温度分布模型,修改初始条件为:环境温度为80ºC,确定出合适的$\uppercase\expandafter{\romannumeral2}$层和$\uppercase\expandafter{\romannumeral4}$层的厚度,使得确保工作60分钟时,假人皮肤外侧温度不超过$47^{\circ}C$,且超过$44^{\circ}C$的时间不超过5分钟。

\newpage

\section{模型描述}
    \indent 首先,我们根据热扩散率的公式$k=\frac{\lambda}{C\rho}$可以计算得到各层的热扩散系数$k$,得到下表格:\\
% Table generated by Excel2LaTeX from sheet 'Sheet1'
\begin{table}[htbp]
    \centering
      \begin{tabular}{|c|c|c|c|c|r|}
      \hline
      \multicolumn{6}{|c|}{专用服装材料的参数值} \bigstrut\\
      \hline
      分层   & \multicolumn{1}{p{5em}|}{密度\newline{}($kg/m^{3}$)} & \multicolumn{1}{p{5em}|}{比热\newline{}($J/(kg\cdot^{\circ}C)$)} & \multicolumn{1}{p{5em}|}{热传导率\newline{}($W/(m\cdot^{\circ}C)$)} & \multicolumn{1}{p{5em}|}{厚度\newline{}($mm$)} & \multicolumn{1}{p{5em}|}{热扩散系数($m^{2}/s$)} \bigstrut\\
      \hline
      I层    & 300   & 1377  & 0.082 & 0.6   & $1.985\times10^{-7}$ \bigstrut\\
      \hline
      II层   & 862   & 2100  & 0.37  & 0.6-25 & $2.044\times10^{-7}$ \bigstrut\\
      \hline
      III层  & 74.2  & 1726  & 0.045 & 3.6   & $3.5137\times10^{-7}$ \bigstrut\\
      \hline
      IV层   & 1.18  & 1005  & 0.028 & 0.6-6.4 & $2.3611\times10^{-5}$ \bigstrut\\
      \hline
      \end{tabular}%
      \caption{专用服装材料的参数值表}
    \label{tab:addlabel}%
  \end{table}%

  \indent 根据热传导方程:
  \begin{equation}
      \centering
      \frac{\partial u}{\partial t}=k\frac{\partial^{2}u}{\partial^{2}s}
  \end{equation}
  \indent 我们通过换元,令$s=\sqrt{k}x$,即$x=sk^{-\frac{1}{2}}$,得到:
  \begin{equation}
      \centering
      \frac{\partial u}{\partial t}=\frac{\partial^{2}u}{\partial^{2}x}
  \end{equation}
  \indent 我们有条件如下:
  \begin{equation}  
    \centering
    u(x,0)=\left\{  
     \begin{array}{lr}  
        T_{1}\quad0<x<\alpha a \\   
        T_{2}\quad\alpha a<x<a 
     \end{array}  
    \right.
\end{equation} 
    \indent 以及边界条件如下:
    \begin{equation}
        \centering
        \left\{  
     \begin{array}{lr}  
        u(0,t)=T_{1} \\   
        u(a,t)=T_{2}
     \end{array}  
    \right.
    \end{equation}
  
    \indent 根据微分方程理论,方程$(1)$的通解具有如下形式:
    \begin{equation}
        \centering
        \left\{  
     \begin{array}{lr}  
        (c_{1}sin(\omega x)+c_{2cos(\omega x)})e^{-\omega^{2}t} \\   
        d_{1}x+d_{2} \\
        (c_{1}e^{\omega x}+c_{2}e^{-\omega x})e^{\omega^{2}t}
     \end{array}  
    \right.
    \end{equation}
    
    \indent 根据边界条件,选择$u(x,t)$的形式为:
    \begin{equation}
        u(x,t)=\frac{T_{2}-T_{1}}{a}x+T_{1}+\sum_{n=1}^{\infty}A_{n}sin(\frac{n\pi}{a}x)e^{-\frac{n^{2}\pi^{2}}{a^{2}}t}
    \end{equation}
    \indent  代入边界条件,边界条件自动满足,对于初始条件,我们有:
    \begin{equation}
        \centering
        \int_{0}^{a}u(x,0)sin(\frac{n\pi}{a}x)dx=\frac{a}{2}An=\int_{0}^{a}sin(\frac{n\pi}{a}x)(\sum_{n=1}^{\infty}A_{n}sin(\frac{n\pi}{a}x)e^{-\frac{n^{2}\pi^{2}}{a^{2}}t})dx
    \end{equation}
    \indent 利用python,我们最终得到,参数如下: 
    \begin{equation}
        \centering
        \left\{  
     \begin{array}{lr}  
        \alpha=0.75 \\   
        a=47.380508849 \\
        x=13.815158896
     \end{array}  
    \right.
    \end{equation}

    
    
     








\end{document} 